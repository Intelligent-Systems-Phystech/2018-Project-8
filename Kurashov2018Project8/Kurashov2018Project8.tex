\documentclass[12pt,twoside]{article}
\usepackage{jmlda}
%\NOREVIEWERNOTES
\title
    [Порождение признаков с помощью локально-аппроксимирующих моделей.]
    {Порождение признаков с помощью локально-аппроксимирующих моделей.}
\author
    [Курашов~И.\,М.]
    {Гальцева~А., Гильмутдинов~Н.\,И., Горностаев~А.\,А., Курашов~И.\,М., Мулюков~А.\,Р., Рябов~А., Спивак~В.\,С.}
\thanks
    {Научный руководитель: Нейчев~Р.\,Г.
    Задачу поставили: Нейчев~Р.\,Г., Стрижов~В.\,В.
    Консультанты: Нейчев~Р.\,Г., Терехов~О.}

\organization
    {Московский физико-технический институт}
\abstract
    {Статья посвящена исследованию проблемы синтезации признаков с использованием локально-аппроксимирующий моделей. В работе проверяется корректность применения гипотезы о простоте выборки для порожденных признаков. Также внимание уделено оценке информативности параметров аппроксимирующих моделей. Рассматриваются методы определения вида деятельности человека по измерениям акселерометра и гироскопа. В контексте данной работы предполагается кластеризация элементарных движений в пространстве описаний временных рядов.

\bigskip

\textbf{Ключевые слова}: \emph{ временной ряд, локально-аппроксимирующая модель}.}

\begin{document}
\maketitle

\section{Введение}
Работа посвящена поиску оптимальных признаков для задачи классификации видов деятельности человека. Исследование проводится с целью автоматизации порождения признаков слабоструктуированных данных, таких как временные ряды. Оптимальный выбор признаков должен удовлетворять выборкам временных рядов с различными частотами. Также предлагаемый в данной работе метод должен обеспечивать минимальное расхождение в точности задачи классификации с различными множествами ответов.

Проблема оптимального порождения признаков решается множеством способов: в работе \cite{journals/titb/MotrenkoS16} выделяются фундаментальные периоды временных рядов, в \cite{journals/titb/KarStr16} внимание уделено сегментации временного ряда различными способами. В данной работе задача решается с помощью построения локально-аппроксимирующих моделей исходной выборки. Предлогаемый метод не дает наилучшую точность среди уже имеющихся спрособов , однако является универсальным для данных с различными параметрами выборок.

Исследование проводится на данных временных рядов акселерометра WISDM с целью решения задачи классификации.



\bibliographystyle{plain}
\bibliography{Project8}




\end{document}