\documentclass[12pt,twoside]{article}
\usepackage{jmlda}
%\NOREVIEWERNOTES
\title
    [Порождение признаков с помощью локально-аппроксимирующих моделей.]
    {Порождение признаков с помощью локально-аппроксимирующих моделей.}
\author
    [Курашов~И.\,М.]
    {Гальцева~А., Гильмутдинов~Н.\,И., Горностаев~А.\,А., Курашов~И.\,М., Мулюков~А.\,Р., Рябов~А., Спивак~В.\,С.}
\thanks
    {Научный руководитель: Нейчев~Р.\,Г.
    Задачу поставили: Нейчев~Р.\,Г., Стрижов~В.\,В.
    Консультанты: Нейчев~Р.\,Г., Терехов~О.}

\organization
    {Московский физико-технический институт}
\abstract
    {Статья посвящена исследованию проблемы синтезации признаков с использованием локально-аппроксимирующий моделей. В работе проверяется корректность применения гипотезы о простоте выборки для порожденных признаков. Также внимание уделено оценке информативности параметров аппроксимирующих моделей. Рассматриваются методы определения вида деятельности человека по измерениям акселерометра и гироскопа. В контексте данной работы предполагается кластеризация элементарных движений в пространстве описаний временных рядов.

\bigskip

\textbf{Ключевые слова}: \emph{ временной ряд, локально-аппроксимирующая модель}.}

\begin{document}
\maketitle

\section{Введение}
Работа посвящена поиску оптимальных признаков для слабоструктуированной задачи. В частности, рассматривается набор временных рядов. Для нахождения признаков используются модели, локально аппроксимирующие элементы выборки.
%(что за модели?)
Решается вспомогательная задача приведения рядов с различными частотами к одинаковым промежуткам разбиения.
%(как это связано с предыдущим?)
Также внимание уделено устойчивости моделей на различных данных. (будет дополнено)
%(что-то нужно добавить)



\nocite{journals/titb/MotrenkoS16}
\nocite{journals/titb/KarStr16}

\bibliographystyle{plain}
\bibliography{Project8}




\end{document}